% Generated by GrindEQ Word-to-LaTeX 
\documentclass{article} % use \documentstyle for old LaTeX compilers

\usepackage[utf8]{inputenc} % 'cp1252'-Western, 'cp1251'-Cyrillic, etc.
\usepackage[english]{babel} % 'french', 'german', 'spanish', 'danish', etc.
\usepackage{amsmath}
\usepackage{amssymb}
\usepackage{txfonts}
\usepackage{mathdots}
\usepackage[classicReIm]{kpfonts}
\usepackage[pdftex]{graphicx}

% You can include more LaTeX packages here 


\begin{document}

%\selectlanguage{english} % remove comment delimiter ('%') and select language if required


\noindent Universidad Mariano G\'{a}lvez de Guatemala

\noindent ingenier\'{i}a en sistemas

\noindent Seguridad y Auditoria de sistemas

\noindent Proyecto Final: Implementaci\'{o}n de PFSENSE 

\noindent 

\noindent Luis Fernando Puluc Barrios 
\[7690-16-5181\] 


\noindent Cristian Alejandro G\'{o}mez P\'{e}rez
\[7690-10-9778\] 


\noindent 

\noindent 

\noindent 

\noindent 

\noindent 

\noindent Para la implementaci\'{o}n de pfsense es necesario la creaci\'{o}n de una VPC la cual configuraremos de la siguiente manera. 

\noindent En el men\'{u} izquierdo seleccionamos Your VPCs y procedemos a crearla.

\noindent 

\noindent 

\noindent Procedemos a ingresar los datos, la cual creamos como pesense\_vpc con los siguientes datos.

\noindent 

\noindent 

\noindent \includegraphics*[width=6.19in, height=3.30in, trim=0.02in 0.15in 1.07in 0.00in]{image1}

\noindent 

\noindent 

\noindent 

\noindent 

\noindent Es necesario crear el Peering Connectios  para que este apunte a nuestro end-point. 

\noindent . 

\noindent 

\noindent \includegraphics*[width=6.60in, height=3.09in, trim=0.00in 0.48in 1.18in 0.11in]{image2}

\noindent 

\noindent 

\noindent Prcedemos a dar clic en cr\'{e}ate customer Gateway luego ingresaremos el BGP ASN este es el sistema autono del Peering Connectios  

\noindent 

\noindent 

\noindent \includegraphics*[width=6.05in, height=3.57in, trim=0.00in 0.66in 2.72in 0.00in]{image3}

\noindent 

\noindent 

\noindent 

\noindent Es necesario crear Subnets, ya por defecto AWS nos proporciona una cantidad de subnet por default, procedemos a agregar una nueva. 

\noindent 

\noindent \includegraphics*[width=6.43in, height=2.78in, trim=0.02in 0.16in 0.08in 0.13in]{image4}

\noindent 

\noindent 

\noindent 

\noindent 

\noindent 

\noindent 

\noindent 

\noindent Procedemos a crear la nueva subnet. 

\noindent 

\noindent 

\noindent \includegraphics*[width=6.66in, height=1.00in, trim=0.00in 2.05in 0.00in 0.11in]{image5}

\noindent 

\noindent 

\noindent 

\noindent 

\noindent 

\noindent 

\noindent 

\noindent 

\noindent 

\noindent 

\noindent 

\noindent 

\noindent 

\noindent Ingresamos el  rango de IP a utilizar y procedemos a crear la subnet.  

\noindent 

\noindent 

\noindent \includegraphics*[width=5.67in, height=3.38in, trim=0.00in 0.00in 1.65in 0.00in]{image6}

\noindent 

\noindent 

\noindent Es necesario crear la Gateway private, esta se utilizara para dirigier el trafico de nuestro Gateway private a nuestro VPC.

\noindent 

\noindent \includegraphics*[width=6.06in, height=3.77in, trim=0.00in 2.43in 6.88in 0.00in]{image7}

\noindent 

\noindent Al momento de crearla es necesario asignarle en ASN default de aws, este permitir\'{a} que tome el sistema aut\'{o}nomo de aws. 

\noindent 

\noindent 

\noindent \includegraphics*[width=6.70in, height=3.01in, trim=0.00in 0.27in 0.30in 0.00in]{image8}

\noindent 

\noindent Luego de haberla creado es necesario a\~{n}adirla a nuestra VPC

\noindent 

\noindent \includegraphics*[width=6.42in, height=2.90in, trim=0.00in 0.31in 0.38in 0.08in]{image9}

\noindent 

\noindent 

\noindent \includegraphics*[width=6.75in, height=3.68in, trim=0.00in 0.77in 2.83in 0.00in]{image10}

\noindent 

\noindent 

\noindent 

\noindent \includegraphics*[width=6.76in, height=3.49in, trim=0.00in 0.24in 1.21in 0.00in]{image11}

\noindent 

\noindent 

\noindent 

\noindent 

\noindent 

\noindent 

\noindent La VPC ya fue asignada a nuestra Gateway. 

\noindent 

\noindent \includegraphics*[width=6.59in, height=3.05in, trim=0.00in 1.12in 2.20in 0.00in]{image12}

\noindent 

\noindent Es necesario realizar la degeneraci\'{o}n de BGP sobre la tabla de ruteo. 

\noindent 

\noindent Precedemos a activarlo. 

\noindent 

\noindent \includegraphics*[width=6.21in, height=2.62in, trim=0.00in 0.97in 1.32in 0.00in]{image13}

\noindent 

\noindent 

\noindent 

\noindent \includegraphics*[width=6.70in, height=2.88in, trim=0.02in 0.25in 0.10in 0.10in]{image14}

\noindent 

\noindent Procedemos a habilitar la propagaci\'{o}n

\noindent 

\noindent 

\noindent \includegraphics*[width=6.61in, height=3.02in, trim=0.00in 0.76in 1.34in 0.00in]{image15}

\noindent 

\noindent 

\noindent Habilitamos con clic la propagaci\'{o}n y procedemos a guardar. 

\noindent 

\noindent 

\noindent 

\noindent 

\noindent 

\noindent 

\noindent 

\noindent 

\noindent 

\noindent Es necesario crear la VPN site-to-site, por lo que procedemos a crearla.  

\noindent 

\noindent \includegraphics*[width=5.60in, height=3.11in, trim=0.00in 0.00in 0.97in 0.00in]{image16}

\noindent 

\noindent Ingresamos los paremetros necesarios con las configuraciones anteriormente creadas y los datos que dejaremos en blanco tomaran los datos predefinidos de AWS

\noindent 

\noindent 

\noindent \includegraphics*[width=5.80in, height=3.34in, trim=0.00in 0.11in 1.47in 0.00in]{image17}

\noindent 

\noindent 

\noindent 

\noindent 

\noindent 

\noindent Como podemos ver AWS nos asigna de forma autom\'{a}tica los segmentos. 

\noindent 

\noindent \includegraphics*[width=6.73in, height=3.91in, trim=0.00in 0.10in 1.88in 0.00in]{image18}

\noindent 

\noindent 

\noindent 

\noindent \includegraphics*[width=6.56in, height=3.92in, trim=0.00in 0.16in 2.05in 0.00in]{image19}

\noindent 

\noindent \includegraphics*[width=6.62in, height=2.95in, trim=0.00in 0.83in 1.54in 0.00in]{image20}

\noindent 

\noindent 

\noindent Luego de proceder a guardar validamos su creaci\'{o}n correctamente. 

\noindent 

\noindent 

\noindent \includegraphics*[width=6.13in, height=1.91in, trim=0.00in 1.21in 0.53in 0.00in]{image21}

\noindent 

\noindent 

\noindent 

\noindent 

\noindent 

\noindent 

\noindent 

\noindent 

\noindent 

\noindent 

\noindent 

\noindent 

\noindent 

\noindent 

\noindent 

\noindent Para realizar la configuraci\'{o}n de PFSENSE AWS nos facilita las configuraciones en un TXT el cual trae todos los par\'{a}metros para la configuraci\'{o}n de pfsense

\noindent 

\noindent 

\noindent 

\noindent \includegraphics*[width=6.54in, height=3.02in, trim=0.00in 0.98in 1.94in 0.00in]{image22}

\noindent 

\noindent 

\noindent 

\noindent \includegraphics*[width=6.71in, height=3.51in, trim=0.00in 0.23in 1.17in 0.00in]{image23}

\noindent 

\noindent 

\noindent 

\noindent 

\noindent Luego de guardarlos procedemos a realizar la configuraci\'{o}n de pfsense con las configuraciones necesarias locales y luego la de aws de la siguiente manera.

\noindent 

\noindent Validando el archivo descargado de aws

\noindent 

\noindent \includegraphics*[width=6.05in, height=4.66in]{image24}

\noindent 

\noindent Realizamos las siguiente configuraciones con PFSENSE

\noindent 

\noindent 

\noindent 

\noindent 

\noindent 

\noindent 

\noindent 

\noindent 

\noindent 

\noindent 

\noindent 

\noindent 

\noindent 

\noindent Para que PFSENSE funcione con AWS y las configuraciones coincidan con las realizadas con AWS se debe de configurar de la siguiente manera con los siguientes par\'{a}metros. 

\noindent 

\noindent \includegraphics*[width=5.60in, height=3.32in, trim=0.82in 0.00in 0.62in 0.00in]{image25}

\noindent 

\noindent 

\noindent 

\noindent \includegraphics*[width=5.67in, height=3.65in, trim=1.04in 0.00in 0.91in 0.00in]{image26}

\noindent 

\noindent 

\noindent 

\noindent 

\noindent 

\noindent \includegraphics*[width=6.01in, height=3.56in, trim=1.11in 0.27in 0.84in 0.00in]{image27}

\noindent 

\noindent 

\noindent \includegraphics*[width=6.17in, height=4.48in, trim=1.72in 0.03in 1.80in 0.26in]{image28}

\noindent \includegraphics*[width=5.90in, height=4.25in, trim=1.01in 0.00in 0.69in 0.00in]{image29}

\noindent 

\noindent 

\noindent 

\noindent 

\noindent 

\noindent 

\noindent 

\noindent 

\noindent 

\noindent 

\noindent 

\noindent 

\noindent 

\noindent 

\noindent 

\noindent 

\noindent 

\noindent 

\noindent 

\noindent 

\noindent 

\noindent 

\noindent 

\noindent 

\noindent Luego con PFSENSE ya configurado ingresamos los datos obtenidos de la VPN de AWS: 

\noindent \includegraphics*[width=6.59in, height=3.35in, trim=0.16in 0.87in 0.02in 0.30in]{image30}

\noindent 

\noindent 

\noindent \includegraphics*[width=6.41in, height=4.80in, trim=0.33in 0.00in 0.44in 0.00in]{image31}

\noindent 

\noindent 

\noindent 

\noindent 

\noindent 

\noindent 

\noindent \includegraphics*[width=6.75in, height=3.85in, trim=0.15in 0.46in 0.17in 0.41in]{image32}

\noindent 

\noindent Y prodedemos a guardar 

\noindent 

\noindent Para validar que nuestro PFSENSE ya este capturando el trafico verificaremos su estatus.  

\noindent 

\noindent 

\noindent \includegraphics*[width=6.22in, height=2.09in, trim=0.42in 0.56in 0.09in 0.33in]{image33}

\noindent 

\noindent Como podemos ver en Data ya esta controlando el trafico de nuestra red virtual de AWS.

\noindent 

\noindent 

\noindent \includegraphics*[width=6.14in, height=1.71in, trim=1.04in 1.78in 1.11in 0.38in]{image34}


\end{document}

